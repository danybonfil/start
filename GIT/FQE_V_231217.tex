% Options for packages loaded elsewhere
\PassOptionsToPackage{unicode}{hyperref}
\PassOptionsToPackage{hyphens}{url}
%
\documentclass[
]{article}
\usepackage{amsmath,amssymb}
\usepackage{iftex}
\ifPDFTeX
  \usepackage[T1]{fontenc}
  \usepackage[utf8]{inputenc}
  \usepackage{textcomp} % provide euro and other symbols
\else % if luatex or xetex
  \usepackage{unicode-math} % this also loads fontspec
  \defaultfontfeatures{Scale=MatchLowercase}
  \defaultfontfeatures[\rmfamily]{Ligatures=TeX,Scale=1}
\fi
\usepackage{lmodern}
\ifPDFTeX\else
  % xetex/luatex font selection
\fi
% Use upquote if available, for straight quotes in verbatim environments
\IfFileExists{upquote.sty}{\usepackage{upquote}}{}
\IfFileExists{microtype.sty}{% use microtype if available
  \usepackage[]{microtype}
  \UseMicrotypeSet[protrusion]{basicmath} % disable protrusion for tt fonts
}{}
\makeatletter
\@ifundefined{KOMAClassName}{% if non-KOMA class
  \IfFileExists{parskip.sty}{%
    \usepackage{parskip}
  }{% else
    \setlength{\parindent}{0pt}
    \setlength{\parskip}{6pt plus 2pt minus 1pt}}
}{% if KOMA class
  \KOMAoptions{parskip=half}}
\makeatother
\usepackage{xcolor}
\usepackage[margin=1in]{geometry}
\usepackage{graphicx}
\makeatletter
\def\maxwidth{\ifdim\Gin@nat@width>\linewidth\linewidth\else\Gin@nat@width\fi}
\def\maxheight{\ifdim\Gin@nat@height>\textheight\textheight\else\Gin@nat@height\fi}
\makeatother
% Scale images if necessary, so that they will not overflow the page
% margins by default, and it is still possible to overwrite the defaults
% using explicit options in \includegraphics[width, height, ...]{}
\setkeys{Gin}{width=\maxwidth,height=\maxheight,keepaspectratio}
% Set default figure placement to htbp
\makeatletter
\def\fps@figure{htbp}
\makeatother
\setlength{\emergencystretch}{3em} % prevent overfull lines
\providecommand{\tightlist}{%
  \setlength{\itemsep}{0pt}\setlength{\parskip}{0pt}}
\setcounter{secnumdepth}{5}
\newlength{\cslhangindent}
\setlength{\cslhangindent}{1.5em}
\newlength{\csllabelwidth}
\setlength{\csllabelwidth}{3em}
\newlength{\cslentryspacingunit} % times entry-spacing
\setlength{\cslentryspacingunit}{\parskip}
\newenvironment{CSLReferences}[2] % #1 hanging-ident, #2 entry spacing
 {% don't indent paragraphs
  \setlength{\parindent}{0pt}
  % turn on hanging indent if param 1 is 1
  \ifodd #1
  \let\oldpar\par
  \def\par{\hangindent=\cslhangindent\oldpar}
  \fi
  % set entry spacing
  \setlength{\parskip}{#2\cslentryspacingunit}
 }%
 {}
\usepackage{calc}
\newcommand{\CSLBlock}[1]{#1\hfill\break}
\newcommand{\CSLLeftMargin}[1]{\parbox[t]{\csllabelwidth}{#1}}
\newcommand{\CSLRightInline}[1]{\parbox[t]{\linewidth - \csllabelwidth}{#1}\break}
\newcommand{\CSLIndent}[1]{\hspace{\cslhangindent}#1}
\usepackage{setspace}
\usepackage{float}
\floatplacement{figure}{H}
\ifLuaTeX
  \usepackage{selnolig}  % disable illegal ligatures
\fi
\IfFileExists{bookmark.sty}{\usepackage{bookmark}}{\usepackage{hyperref}}
\IfFileExists{xurl.sty}{\usepackage{xurl}}{} % add URL line breaks if available
\urlstyle{same}
\hypersetup{
  pdfauthor={Daniel Bonfil},
  hidelinks,
  pdfcreator={LaTeX via pandoc}}

\title{\hfill\break
\hfill\break
Affective Judgements in Retrospective Voting:\\
The Role of Framing and the End Bias\\
\strut \\}
\usepackage{etoolbox}
\makeatletter
\providecommand{\subtitle}[1]{% add subtitle to \maketitle
  \apptocmd{\@title}{\par {\large #1 \par}}{}{}
}
\makeatother
\subtitle{\hfill\break
\hfill\break
FGV EBAPE\\
\strut \\
\strut \\
First Qualifying Examination\\
(Conceptual Paper)\\
\strut \\}
\author{Daniel Bonfil}
\date{2023-12-17}

\begin{document}
\maketitle

{
\setcounter{tocdepth}{2}
\tableofcontents
}
\newpage
\doublespacing

\hypertarget{abstract}{%
\subsection{Abstract}\label{abstract}}

This research explores how affective judgments---emotional ties and
sentiments towards political leaders---influence citizens' economic
evaluations of identified incumbents, particularly in retrospective
voting. Affective judgments can shape perceptions of political leaders,
influencing performance evaluations by associating leaders with good or
bad times, regardless of their direct responsibility. Retrospective
voting often involves assessing incumbents' past economic performance.
However, the end bias---voters' overemphasis on the last year of
economic growth relative to the previous three in a presidential
term---can distort their judgments. Previous work by Healy \& Lenz
(2014) addressed the end bias by emphasizing access to ``right''
information through yearly and cumulative framing. However, this
research challenges these conclusions, moving beyond hypothetical
scenarios to examine real economic data and the interplay of affect and
cognitive biases. Our central question is whether framing---the
selection and presentation of---economic information with both yearly
and cumulative levels can mitigate the influence of affective judgments,
facilitating a more balanced and informed assessment of incumbents'
economic performance. Through a survey experiment, we employ a sentiment
thermometer to measure participants' affective responses to presidents
and how framing conditions may lead to unintentional emphasis on
election-year economic growth rather than evaluating leaders evenly
throughout their terms.

\hypertarget{introduction}{%
\subsection{Introduction}\label{introduction}}

Retrospective voting, the process of assessing incumbents' past
performance, is particularly susceptible to the influence of biases,
unrelated events, and framing (Huber et al., 2012). Conventional
defenses of democratic accountability rely on the theory of
retrospective voting to suggest that citizens would behave rationally
based on recollections of incumbents' past performance, even in the
presence of limited information (Fiorina, 1981; Key, 1966). However,
based on Kramer (1971), Fair (1978), and Tufte (1978), research by Achen
\& Bartels (2004) challenges this notion, arguing that voters often make
retrospective voting decisions based on how they feel about recent
conditions and their susceptibility to manipulation by governments.

The evaluation of presidential performance plays a controversial role in
retrospective voting especially due to the inherent limitations of human
information-processing capacity (Simon, 1957) and the widespread use of
cognitive heuristics or mental shortcuts (Kahneman, 2011; Lau \&
Redlawsk, 1997). Affective judgments, or emotional ties and sentiments
towards political leaders, have been shown to shape perceptions of
political figures, influencing performance evaluations by associating
leaders with good or bad times, regardless of their direct
responsibility (Campello \& Zucco, 2022; Lodge \& Taber, 2013). In the
context of economic performance evaluations, the end bias, or the
emphasis on election-year economic conditions, can lead voters to
overweigh the last year of economic growth relative to the previous
three in a presidential term (Healy \& Lenz, 2014). These findings
suggest that affective judgments can play a significant role in
influencing retrospective voting decisions, leading to potentially
unfair and inaccurate economic performance evaluations of incumbents.

Previous research by Healy \& Lenz (2014), examined the role of framing,
or the selection and presentation of information, in mitigating the end
bias, a tendency for voters to weigh recent economic conditions more
heavily when evaluating incumbents. Their findings suggest that framing
economic information with both yearly and cumulative levels can reduce
the end bias, indicating that framing can help voters make more balanced
and informed retrospective assessments. However, this research used
hypothetical economic data and has not specifically addressed the
influence of affective judgments on retrospective voting.

Along with framing economic information, it is crucial to understand the
influence of affective judgments in evaluating identified incumbents.
Campello \& Zucco (2022) emphasize the issue of misattribution of
responsibility, which occurs when voters attribute economic conditions
to incumbents even when those conditions are determined by factors
beyond governmental control. They show that misattribution is not solely
a matter of access to information, but rather that voters often struggle
to accurately apply available information. They find that individuals'
prior affective ties to political leaders cloud their willingness and
capacity to discount exogenous conditions and therefore to update their
judgments of political figures.

In addition, Campello \& Zucco (2022) find that individuals are more
likely to use information to discount exogenous conditions and update
their judgments, particularly in the case of a hypothetical rather than
an identified incumbent. Moreover, their findings imply that updates are
even less likely when information is provided only one time as in their
experimental setting, which reinforces the pessimistic perspective of
Huber et al. (2012) on the effectiveness of information provision in
modifying individual behavior. These arguments contrast with Lodge \&
Taber (2013), who suggest that people automatically update their
attitudes toward a variety of social and political objects at the time
they encounter relevant information.

Affective ties pose a barrier to the objective evaluation of available
information and the updating of judgments especially when they clash
with affect for political figures (Campello \& Zucco, 2022). However,
the magnitude of exogenous factors varies significantly among regions,
generating positive or negative affective judgments that may be more
difficult to reverse in some countries compared to others. Notably, the
study by Healy \& Lenz (2014) focuses on the USA while research by
Campello \& Zucco (2022) is based in Brazil, where external factors
might exert a stronger influence on affective judgments.

Campello \& Zucco (2020) claim that exposure to exogenous shocks,
strongly associated with dependence on commodities and variations in
trade, is a fundamental structural difference between developed and
developing economies. They argue that countries highly exposed to
volatile exogenous conditions face more challenges in identifying the
competence of governments based on the economy. Finally, they suggest
that misattribution appears not to be just a problem of information
provision but rather a cognitive issue, and that the processing of
relevant information could allow for a better attribution of
responsibility for the economy in developed countries than in developing
ones.

The present research aims to build on these findings by investigating
whether framing economic information with both yearly and cumulative
levels can mitigate the influence of affective judgments, facilitating a
more balanced and informed assessment of incumbents' economic
performance. As Lodge \& Taber (2013) argue, affective judgments are
activated faster than any other consideration, and often operate
unconsciously. This means that voters may have already formed affective
judgments about an incumbent's economic performance based on
recollections of particularly good or bad economic conditions, without
fully considering the incumbent's overall record. Moreover, Taber and
Lodge indicate that affective judgments are closely linked to long term
memory (LTM). This means that once voters have formed an affective
judgment about an incumbent, it can be very difficult to change their
minds, even in the face of contrary evidence.

Heuristics, or mental shortcuts, aid individuals to process information
and improve decision-making, especially due to the real-time limitations
of conscious processing (Redlawsk \& Lau, 2013; Taber \& Young, 2013).
While adaptive and efficient, they can also lead to biases and errors in
judgment (Kuklinski \& Quirk, 2000; Tversky \& Kahneman, 1974). For
instance, the ``end heuristic'' can lead to the end bias in various
contexts from undergoing colonoscopies (Redelmeier \& Kahneman, 1996) to
watching TV ads (Baumgartner et al., 1997). Similarly, the ``affect
heuristic,'' also referred to as the ``likeability heuristic,''
automatically links positive or negative emotions to familiar social
objects in LTM, potentially leading to motivated reasoning. Motivated
reasoning is a rationalization process driven by unconscious affective
biases, such as disconfirmation bias and active counterarguing (Taber \&
Young, 2013). Consequently, our research aims to determine whether the
findings by Healy \& Lenz (2014) which, in the context of economic
evaluations where framing can mitigate the influence of the end bias,
hold in the presence of affective judgments.

Lau \& Redlawsk (1997) suggest that the use of cognitive heuristics
seems a reasonable strategy to the extent that it helps align voters'
interests and values with candidates' positions and attributes. By
presenting voters with a more comprehensive picture of the incumbent's
economic performance, we hope to override the influence of affective
judgments and encourage voters to make more balanced and informed
evaluations. To test this hypothesis, we will conduct a survey
experiment where we will randomly assign participants to one of four
experimental conditions. Participants in the identified conditions will
first complete a sentiment thermometer to measure their affective
judgments towards each president included in the experiment. This will
allow us to assess whether affective judgments are driving their
evaluations of the incumbent's economic performance.

The proposed framing manipulation aligns with the affect-driven
dual-process theory of motivated reasoning proposed by Lodge \& Taber
(2013). This theory suggests that voters have two distinct modes of
thinking: an unconscious, automatic mode ``System 1'' and a conscious,
deliberative mode ``System 2''. Affective judgments are more likely to
influence voters when they are thinking in automatic mode. By framing
economic growth in a way that engages the conscious, deliberative mode,
we hope that participants use and process information to evaluate
incumbents even in the presence of prior affective judgements and the
challenges to update them.

The first manipulation for framing will involve presenting participants
with real economic data framed at either the yearly level (control
condition) or both the yearly and cumulative levels (treatment
condition). This manipulation will allow us to examine whether framing
economic information in a way that emphasizes both yearly and cumulative
performance helps participants to consider the entire term of an
incumbent, rather than just the election year. The second manipulation
will involve presenting participants with real economic data for either
an unidentified incumbent (blind condition) or an identified incumbent
(identified condition). This manipulation will allow us to assess the
impact of affective judgments on participants' evaluations.

In the blind conditions, we expect to find that participants in the
treatment condition will have more accurate and fair evaluations of the
incumbent's economic performance than participants in the control
condition. This is because framing the economic information in a way
that emphasizes both yearly and cumulative performance will help them to
consider the entire term of the incumbent, rather than just the election
year.

In the identified conditions, we expect that participants in the
treatment condition will be less likely to be influenced by their
affective judgments of the incumbent than participants in the control
condition. This is because the additional information provided in the
treatment condition will help them to make a more objective assessment
of the incumbent's economic performance. Finally, we expect that the
treatment effect will be larger (smaller) among participants who hold
weak (strong) positive or negative affective judgments of the incumbent.
This backlash is because individuals with stronger affective judgments
are more likely to rationalize or resist updating information when
making evaluations, whereas those with weaker affective judgements may
be more receptive to new information.

This research has important implications for our understanding of how
citizens evaluate political leaders and make voting decisions. It
suggests that framing economic information could be a valuable tool for
reducing the influence of affective judgments and promoting more
accurate and fair evaluations of incumbents. This research also has
implications for the design of news media campaigns and for the
development of more effective communication strategies by government
statistical agencies.

The rest of this article is structured as follows. Section (2)
introduces notions of bounded rationality, heuristics, the surge of
dual-process models, and framing. Alongside, we explain the relationship
between the use of yearly and cumulative economic information, the end
bias, and the implications of framing and affective judgements. Section
(3) describes the data and experimental design, followed by the
hypotheses, and expected results in Section (4). Next, we interpret and
explain the main results of the online experiments in Section (5)
{[}TBD{]}, followed by the implications, limitations, and future
research possibilities in Section (6) {[}TBD{]}. Finally, concluding
remarks and applications are provided in Section (7) {[}TBD{]}.

\hypertarget{theoretical-framework}{%
\subsection{Theoretical Framework}\label{theoretical-framework}}

\hypertarget{heuristics-and-bounded-rationality}{%
\subsubsection{Heuristics and Bounded
Rationality}\label{heuristics-and-bounded-rationality}}

Traditional economic models of human decision-making have assumed
perfect rationality, portraying individuals as fully informed, perfectly
logical, and utility-maximizing agents (Chong, 2013; Neumann \&
Morgenstern, 2007). However, this idealized view has been challenged by
emerging perspectives that acknowledge the limitations of human
cognition and the influence of psychological factors on decision-making.

Herbert Simon's concept of bounded rationality marked a significant
departure from the traditional notion of perfect rationality.
Recognizing that individuals operate under constraints of information,
time, and processing power, Simon (1957) proposed that decision-making
is often bounded by these limitations, leading to strategies that are
satisficing rather than optimal. This concept challenged the assumptions
of Rational Choice Theory (RCT), suggesting that decisions are often
made using heuristics, mental shortcuts that allow for quick and
efficient decision-making, but can also lead to biases and suboptimal
outcomes.

The paradox of voting, identified by Anthony Downs, provides a
compelling illustration of the shortcomings of rational choice theory.
Downs (1957) argued that the expected benefits of voting, such as
influencing the outcome of an election, are often outweighed by the
costs involved, such as the time and effort required to vote, and inform
oneself about candidates and issues. Despite this, voting remains a
common practice, suggesting that individuals may make decisions based on
factors beyond self-interest and rational calculations.

Behavioral decision theory (BDT) emerged as a response to the
limitations of RCT, recognizing the influence of psychological factors,
such as emotions, cognitive biases, and social influences, on
decision-making (Redlawsk \& Lau, 2013). A central concept in BDT is the
use of heuristics, which allow individuals to navigate complex
decision-making situations and reach satisficing rather than
value-maximizing results without requiring extensive deliberations
(Kahneman, 2003). These heuristics, while often adaptive and efficient,
can lead to systematic biases and errors in judgment (Tversky \&
Kahneman, 1974).

The concept of bounded rationality and the use of heuristics are
particularly relevant in the context of political decision-making. Faced
with an overwhelming amount of information about candidates, policies,
and political issues, voters often rely on heuristics to make informed
choices (Kuklinski \& Quirk, 2000). Heuristics such as party
identification, candidate endorsements, and incumbency advantage can
provide useful shortcuts, allowing voters to make judgments quickly and
efficiently. However, these heuristics can also lead to biases and
suboptimal outcomes. For instance, voters may rely too heavily on party
identification, leading to partisan biases that may not reflect their
preferences (Lau \& Redlawsk, 2001).

Prospect theory, developed by Kahneman \& Tversky (1979), further
highlights the limitations of human decision-making. This theory
suggests that individuals are more sensitive to losses than to gains,
leading to risk-averse behavior in the domain of gains and risk-seeking
behavior in the domain of losses. This bias can explain seemingly
irrational behavior, such as individuals rejecting a fair bet with an
equal chance of winning or losing a small amount of money but accepting
the same bet when the potential loss is framed as a gain.

Despite their potential pitfalls, heuristics play a crucial role in
human decision-making, allowing individuals to cope with information
overload and make choices in complex situations without requiring
extensive deliberation (Lau \& Redlawsk, 2001). In many cases,
heuristics can lead to satisfactory outcomes, particularly when the
costs of obtaining perfect information outweigh the potential benefits.
However, understanding the limitations of heuristics is essential for
making informed decisions. By recognizing the conditions under which
fast thinking, or System 1, can lead to errors, individuals can engage
in more deliberate, System 2 thinking, improving the quality of their
decisions (Kahneman, 2011).

\hypertarget{dual-process-models-and-rationalizers}{%
\subsubsection{Dual-Process Models and
Rationalizers}\label{dual-process-models-and-rationalizers}}

The recognition of the dual-process nature of human decision-making, as
proposed by Kahneman (2003), led to a surge in dual-process models in
various fields, including political psychology. Lodge \& Taber (2013)
adopted this framework to understanding political behavior, highlighting
the interplay between System 1, the fast, intuitive, and automatic
system, and System 2, the slow, deliberate, and effortful system. Within
this framework, the concept of affective judgements, or rooted emotional
associations linked to political objects, play a crucial role in shaping
political preferences.

Moreover, Lodge \& Taber (2013) define the ``hot cognition hypothesis,''
which holds that all cognitive objects are linked to affective tags in
LTM via an associative network, and cannot be activated without
simultaneously activating their affective tags (Lodge \& Taber, 2005).
Motivated reasoning, a pervasive example of hot cognition, suggests that
we are motivated to maintain our existing beliefs and preferences,
sometimes at the expense of objective evaluation (Bartels, 1996). This
tendency can lead to biased interpretations of information, selective
recall of favorable evidence, dismissal of opposing viewpoints, the
overemphasis of confirming evidence, and the clinging to ideological
positions even in the face of overwhelming evidence to the contrary.

The System 1 likeability heuristic, proposed by Lodge \& Taber (2013),
guides candidate evaluations based on stored affective associations.
This heuristic operates through associative pathways, allowing rapid
activation of affective tags that automatically integrate and update
incoming information. The process of motivated reasoning is particularly
pronounced among individuals who have extensively considered political
figures, as they have developed stronger affective associations in LTM.
In contrast, those with less political interest and knowledge may not
have formed such strong affective associations, and therefore, may not
exhibit the same pattern of facilitation and inhibition that indicates
automatic affect. This highlights the role of individual differences in
political engagement in shaping the influence of affective judgments.

Affective judgments, intertwined with motivated reasoning, can cloud
objective assessments of political leaders and policies. Campello \&
Zucco (2022) research demonstrates that misattribution is not solely a
matter of access to information, but rather that voters often struggle
to accurately apply available information and update their judgements.
Specifically, they find that individuals' prior affective ties cloud
their capacity to discount exogenous conditions and hinder the objective
evaluation of available information especially when it clashes with
their affective judgments of political figures. This suggests that
affective judgments can significantly hamper the use of available
information hence, we are interested in whether different framing
conditions can help prevent the impairment of objective evaluation of
information.

\hypertarget{framing-friend-or-foe}{%
\subsubsection{Framing: Friend or Foe?}\label{framing-friend-or-foe}}

Framing, coupled with cognitive biases plays a pivotal role in shaping
citizens' responses to economic information. The nexus between
democratic accountability and framing dynamics holds significance as
political actors and media outlets strategically frame issues to shape
public opinion and perceptions of government performance (McCombs \&
Shaw, 1972). Framing's impact on public discourse and agenda-setting is
further complicated by its interaction with affective judgments,
scholars like Goffman (1986) and Lakoff (2014) argue that the way an
issue or message is framed can activate certain cognitive schemas and
influence how individuals understand and respond to it. Moreover, Marcus
et al. (2000) and Sniderman \& Theriault (2004) delve into how framing
shapes public opinion and behavior, influencing how citizens perceive
their representatives and how they hold them accountable.

Framing, or the selection and presentation of information, can
significantly influence individuals' preferences and choices, even when
the underlying facts remain the same (Druckman, 2001a). As Huber et al.
(2012) demonstrate, subtle framing manipulations can significantly alter
individuals' retrospective assessments. This underscores the potential
of framing to influence the way people evaluate and interpret
information as well as how they react to it.

Equivalence frames, logically equivalent but differently phrased
descriptions of the same issue, can lead to different decisions (Chong,
2013). For instance, the example by Kahneman \& Tversky (1979)
highlights the inconsistency of human decision-making as individuals
reversed their preferences when presented with options framed as gains,
people tended to make risk-averse choices, opting for the safer option
with a guaranteed outcome. However, when the same options were framed as
losses, people made risk-seeking choices, preferring the riskier option
with the potential to avoid a loss, even if it meant a higher chance of
incurring an even greater loss. Our take is that framing can serve as a
tool to address biases and sustain the effectiveness of heuristics.
Healy \& Lenz (2014) provide compelling evidence that even psychological
biases can be overcomed when individuals have access to pertinent
information. Their study reveals that when voters are presented with the
attribute of total economic growth over a president's term, they are
better equipped to evaluate presidents based on their overall
performance, mitigating the influence of the end bias.

As Chong (2013) highlights, framing effects have a profound impact on
how individuals perceive and respond to political issues. However, the
mere presence of framing may not be sufficient to elicit rational
responses from citizens, particularly in the face of affective
judgments. Individuals vary in their susceptibility to framing effects,
influenced by factors such as cognitive ability, strong attitudes, and
the need to justify preferences (Druckman, 2001b). Those with stronger
attitudes or a higher need for cognitive consistency may be less
susceptible to framing.

While scholars have documented voter responsiveness in various contexts,
we eco Healy \& Malhotra (2013) question regarding whether this
responsiveness incentivizes politicians to maximize social welfare. This
prompts a paradigm shift in retrospective voting studies, moving beyond
the question of whether voters respond to government performance to
examining whether they respond in a manner that aligns their actions
with their intentions.

Framing offers a promising approach to counteract biases and enhance the
effectiveness of heuristics in political decision-making. However, the
presence of affective judgments and motivated reasoning requires careful
consideration of framing techniques to ensure that citizens make
balanced and informed choices, ultimately reinforcing democratic
accountability. In the upcoming sections, we offer a comprehensive
overview of our research design, building upon the previously presented
foundational insights and available data. We aim to contribute to the
existing literature and enhance our understanding of this complex
democratic process.

\hypertarget{data-and-experimental-design}{%
\subsection{Data and Experimental
Design}\label{data-and-experimental-design}}

Our study builds upon the original data and experiment conducted by
Healy \& Lenz (2014), which explores the impact of framing yearly and
cumulative RDI growth rates on participants' economic evaluations while
addressing the end bias. We replicate and extend this experiment using
real economic data sourced from the Bureau of Economic Affairs (BEA).
Specifically, we use data from 17 US presidential tenures from 1940 to
2008, instead of the 25 hypothetical 4-year economies. The dataset
includes Real Disposable Income (\(RDI\)) with both yearly and
cumulative growth rates and the names of the respective presidents
during their terms. To delve into the effects of framing economic
information on mitigating the influence of affective judgments we align
with the approach advocated by Campello \& Zucco (2022), enabling us to
examine whether framing can indeed facilitate a more balanced and
informed assessment of the economic performance of identified
incumbents.

Experimental findings, as highlighted by Huber et al. (2012), reveal
participants' inclination to disproportionately emphasize recent
incumbent performance. Over a series of experiments, Healy \& Lenz
(2014) demonstrate that participants tend to overweight the last year's
performance, despite intending to weigh growth equally across all years.
Using the incumbent party's vote margin (\(incmargin\)) in 17 US
elections, they expose a noteworthy correlation (\(\rho = 0.734\)) with
participants' naïve economic evaluations (\(economy\)) and the
Election-Year Real Disposable Income (\(RDI_y\)) growth rate for each of
the four years, as shown in Figure 1 below. Furthermore, they show that
the end bias can be mitigated by providing voters with both yearly and
cumulative economic information. When both types of information are made
equally accessible, the election-year emphasis is eliminated, leading to
a more balanced and informed evaluation of incumbents' economic
performance.

\begin{figure}
\hypertarget{fig:label}{%
\centering
\includegraphics[width=1\textwidth,height=1\textheight]{C:/Users/danbo/OneDrive - Fundacao Getulio Vargas - FGV/FGV/Q3.23/GIT/GIT/incmarg.png}
\caption{Vote Margin vs RDI Growth and Economic
Evaluations}\label{fig:label}
}
\end{figure}

To reinforce the previous plots, Table 1 below provides regression
results for both `inc\_margin' and `economy' with valuable insights into
the relationship between economic evaluations and voting behavior in the
context of 17 elections. To ensure comparability, both the actual vote
margin and economic evaluations were rescaled to vary from 0 to 1. The
coefficients represent the estimated impact of income growth in each
year of presidents' terms on the rescaled variables.

For `inc\_margin\_scaled,' the coefficients indicate the implicit
weights each year of income growth contributes to the incumbent party's
actual vote margin. Notably, the coefficient for `RDI\_4' (Year 4) is
the highest, suggesting that the income growth in the fourth year has
the most substantial positive impact on the vote margin. The p-values
from the hypothesis tests further support the significance of the
coefficients, highlighting that income growth in the fourth year
significantly influences respondents' economic evaluations relative to
each of the previous years, with exception of year 3 that is only
statistically significant at the 10\% level.

Similarly, for `economy\_scaled,' the coefficients indicate the implicit
weights between income growth in each year and respondents' economic
evaluations. `RDI\_4' again stands out with the highest coefficient,
indicating a pronounced impact on economic evaluations. The associated
p-values for the hypothesis tests comparing the coefficients for
`RDI\_4' with other years are all statistically significant, reinforcing
the importance of the fourth year's income growth in influencing voters'
decisions.

\begin{center}
  \begin{table}[!htbp]
\centering
\caption{Economic Evaluations and Incumbent Margin vs RDI Growth}
\begin{tabular}{@{\extracolsep{2pt}}lcccc}
\\[-1.8ex]\hline
\hline \\[-1.8ex]
\\[-1.8ex] & \multicolumn{1}{c}{Economy} & \multicolumn{1}{c}{Vote Margin} & \multicolumn{1}{c}{Economy (Rescaled)} & \multicolumn{1}{c}{Vote Margin (Rescaled)}  \\
\\[-1.8ex] & (1) & (2) & (3) & (4) \\
\hline \\[-1.8ex]
 RDI\textsubscript{1} & -0.127$^{*}$ & -0.032$^{}$ & -0.020$^{*}$ & -0.001$^{}$ \\
& (0.061) & (0.700) & (0.010) & (0.021) \\
 RDI\textsubscript{2} & 0.159$^{**}$ & -0.425$^{}$ & 0.026$^{**}$ & -0.013$^{}$ \\
& (0.061) & (0.701) & (0.010) & (0.021) \\
 RDI\textsubscript{3} & 0.494$^{***}$ & 1.483$^{}$ & 0.080$^{***}$ & 0.044$^{}$ \\
& (0.081) & (0.937) & (0.013) & (0.028) \\
 RDI\textsubscript{4} & 0.812$^{***}$ & 4.368$^{***}$ & 0.131$^{***}$ & 0.129$^{***}$ \\
& (0.093) & (1.070) & (0.015) & (0.032) \\
 const & 2.866$^{***}$ & -8.448$^{**}$ & 0.101$^{*}$ & 0.068$^{}$ \\
& (0.330) & (3.817) & (0.053) & (0.113) \\
\hline \\[-1.8ex]
 Observations & 17 & 17 & 17 & 17 \\
 $R^2$ & 0.916 & 0.640 & 0.916 & 0.640 \\
 Adjusted $R^2$ & 0.888 & 0.520 & 0.888 & 0.520 \\
 Residual Std. Error & 0.634 (df=12) & 7.322 (df=12) & 0.102 (df=12) & 0.216 (df=12) \\
 F Statistic & 32.822$^{***}$ (df=4; 12) & 5.335$^{**}$ (df=4; 12) & 32.822$^{***}$ (df=4; 12) & 5.335$^{**}$ (df=4; 12) \\
\hline
\hline \\[-1.8ex]
\textit{Note:} & \multicolumn{4}{r}{$^{*}$p$<$0.1; $^{**}$p$<$0.05; $^{***}$p$<$0.01} \\
\end{tabular}
\end{table}
\end{center}

The observations from the original experiment suggest a consistent
pattern where the fourth year plays a substantial role in shaping both
the actual vote margin and economic evaluations. The similitudes between
real-world voting behavior and experimental responses enhance the
external validity of the findings, indicating that the experimental
results capture similar dynamics observed in actual elections.

\hypertarget{experimental-conditions}{%
\subsubsection{Experimental Conditions}\label{experimental-conditions}}

The experimental design, as illustrated in Figure 2, will be implemented
in Qualtrics for participants recruited from Amazon Mechanical Turk
(MTurk), drawing from a pool of US citizens. Participants will be
incentivized with a payment of \$0.25 US Dollars for their involvement
in the study. The survey begins with a consent and an attention screener
before engaging in economic evaluations across the different conditions.
Participants in the identified conditions respond a 7-point feeling
thermometer gauging respondents' sentiments toward political leaders.
The survey concludes with a brief demographic, ideological, and party
identification questionnaire.

\begin{figure}
\hypertarget{fig:label}{%
\centering
\includegraphics[width=0.5\textwidth,height=0.5\textheight]{C:/Users/danbo/OneDrive - Fundacao Getulio Vargas - FGV/FGV/Q3.23/GIT/GIT/expdesign2.png}
\caption{Experimental Design}\label{fig:label}
}
\end{figure}

Participants in our study will be randomly assigned to one of four
distinct conditions resulting of two manipulations. The framing
manipulation will involve presenting participants with real economic
data framed at either the yearly level (control condition) or both the
yearly and cumulative levels (treatment condition). This manipulation
will allow us to examine whether framing economic information in a way
that emphasizes both yearly and cumulative performance helps
participants to consider the entire term of an incumbent, rather than
just the election year. The affective manipulation will involve
presenting participants with real economic data for either an
unidentified incumbent (blind condition) or an identified incumbent
(identified condition). This manipulation will allow us to assess the
impact of affective judgments on participants' evaluations.

\begin{itemize}
\tightlist
\item
  In the Framing Manipulation, the Control Condition presents
  participants with yearly growth plots, resembling the top panel of
  Figure 3. In the Treatment Condition it exposed participants to both
  yearly and cumulative growth plots together, including the top and
  bottom panels of Figure 3. We display two presidential terms for
  illustration purposes but participants are only exposed to one term at
  a time (either left or right panels of Figure 3).
\end{itemize}

\begin{figure}
\hypertarget{fig:label}{%
\centering
\includegraphics[width=0.9\textwidth,height=0.9\textheight]{C:/Users/danbo/OneDrive - Fundacao Getulio Vargas - FGV/FGV/Q3.23/GIT/GIT/clintoncarter_cum.png}
\caption{Framing Manipulation}\label{fig:label}
}
\end{figure}

\begin{itemize}
\tightlist
\item
  The Affective Manipulation featured the same graphs but included the
  names of presidents (Identified Condition), as depicted in Figure 4 or
  without names as in Figure 3 (Blind Condition). These conditions
  allowed us to investigate the influence of different framing
  techniques and the impact of presidential identification on
  participants' economic evaluations.
\end{itemize}

\begin{figure}
\hypertarget{fig:label}{%
\centering
\includegraphics[width=0.9\textwidth,height=0.9\textheight]{C:/Users/danbo/OneDrive - Fundacao Getulio Vargas - FGV/FGV/Q3.23/GIT/GIT/clintoncarter_combined.png}
\caption{Affective Manipulation}\label{fig:label}
}
\end{figure}

Following the original experiment, after each plot we will ask ``How
would you rate the condition of the national economy during this period?
Is it very good, fairly good, fairly bad, or very bad?'' Responses will
be recorded to vary from 0 to 10, with 10 corresponding to ``very good''
in order to put RDI growth and economic evaluations on a similar scale.
Please refer to the Annex for details of the online survey.

Figures 3 and 4 illustrate the Real Disposable Income (RDI) yearly
growth rate, clearly depict the findings shown in Figure 1 that, even
when presented with data from all years of a President's Term in a
simple format, most participants placed substantially more weight on the
economy in the final year of presidents' terms. For instance, consider
the left panel, which corresponds to Bill Clinton's initial term. It
portrays a period of moderate average growth, culminating in a notably
strong phase. In contrast, the right panel showcases Jimmy Carter's sole
term, characterized by more robust growth during the first two years,
followed by a marked deceleration in the election year. Interestingly,
despite the cumulative income growth being superior during Carter's
tenure (6.9\% compared to Clinton's 6.2\%), participants consistently
evaluated the economy more favorably during Clinton's first term (6.2
compared to Carter's 3.9 on the 10-point scale). This striking
phenomenon underscores the prevalence of the end bias and its potential
to influence participants' economic evaluations, even when presented
with comprehensive data from all years of a presidential term.

\hypertarget{regression-analysis}{%
\subsubsection{Regression Analysis}\label{regression-analysis}}

To understand the influence of each year on participants' economic
evaluations, we will employ regression analysis. Specifically, we will
regress participants' average economic evaluations on the percentage
change in (\(RDI_y\)) growth for each of the four years within a
presidential term. This regression model is represented by Equation 1:

\[
y_i = \beta_0 + \beta_1 X_1 + \beta_2 X_2 + \beta_3 X_3 + \beta_4 X_4 + \varepsilon_i
\]

Where: - \(y_i\) represents the economic evaluation of participant
\(i\). - \(X_1, X_2, X_3,\) and \(X_4\) represent the (\(RDI_y\)) growth
rates for the four years of the presidential term. -
\(\beta_1, \beta_2, \beta_3,\) and \(\beta_4\) are coefficients
estimating the implicit weight assigned to each year. -
\(\varepsilon_i\) represents the error term.

This regression analysis will be respectively applied on all four
experimental conditions to compare the influence of framing on
participants' evaluations. Additionally, for the two conditions of the
Affective Manipulation, we will perform another analysis including the
interaction term between the Framing and Affect Manipulations, as shown
in the Directed Acyclic Graph (DAG) of Figure 5 below. By replicating
and extending the original experiment with real data and introducing
these experimental conditions, we aim to shed light on the interplay
between framing, cognitive biases, and affective judgments in the
context of retrospective voting.

\begin{figure}
\hypertarget{fig:label}{%
\centering
\includegraphics[width=0.75\textwidth,height=0.75\textheight]{C:/Users/danbo/OneDrive - Fundacao Getulio Vargas - FGV/FGV/Q3.23/GIT/GIT/dag_plot.png}
\caption{DAG: Moderators}\label{fig:label}
}
\end{figure}

\hypertarget{expected-results-and-hypotheses}{%
\subsection{Expected Results and
Hypotheses}\label{expected-results-and-hypotheses}}

In this section, we aim to provide insights into the expected outcomes
of our hypotheses by analyzing Table 2 together with Figure 5 and 6,
which illustrate the effects of framing on participants based on the
original experiment conducted by Healy \& Lenz (2014), which consider 25
hypothetical 4-year economies. In our experimental setup, participants
in the control condition will be exposed to yearly plots, while those in
the treatment condition viewed both yearly and cumulative (\(RDI_y\))
plots of 17 US presidential tenures from 1940 to 2008.

\begin{center}
  \begin{table}[!htbp]
\centering
\caption{Hypothetical Economies Framing Manipulation}

\begin{tabular}{@{\extracolsep{5pt}}lcccc}
\\[-1.8ex]\hline
\hline \\[-1.8ex]
\\[-1.8ex] & \multicolumn{1}{c}{Treatment 0} & \multicolumn{1}{c}{Treatment 1} & \multicolumn{1}{c}{Treatment 0 (Rescaled)} & \multicolumn{1}{c}{Treatment 1 (Rescaled)}  \\
\\[-1.8ex] & (1) & (2) & (3) & (4) \\
\hline \\[-1.8ex]
 RDI\textsubscript{1} & 0.124$^{*}$ & 0.436$^{***}$ & 0.021$^{*}$ & 0.069$^{***}$ \\
& (0.060) & (0.063) & (0.010) & (0.010) \\
 RDI\textsubscript{2} & 0.365$^{***}$ & 0.497$^{***}$ & 0.061$^{***}$ & 0.078$^{***}$ \\
& (0.053) & (0.056) & (0.009) & (0.009) \\
 RDI\textsubscript{3} & 0.462$^{***}$ & 0.545$^{***}$ & 0.077$^{***}$ & 0.086$^{***}$ \\
& (0.058) & (0.061) & (0.010) & (0.010) \\
 RDI\textsubscript{4} & 0.592$^{***}$ & 0.500$^{***}$ & 0.099$^{***}$ & 0.079$^{***}$ \\
& (0.040) & (0.042) & (0.007) & (0.007) \\
 const & 2.591$^{***}$ & 1.792$^{***}$ & -0.124$^{**}$ & -0.184$^{***}$ \\
& (0.286) & (0.302) & (0.048) & (0.048) \\
\hline \\[-1.8ex]
 Observations & 25 & 25 & 25 & 25 \\
 $R^2$ & 0.937 & 0.936 & 0.937 & 0.936 \\
 Adjusted $R^2$ & 0.925 & 0.923 & 0.925 & 0.923 \\
 Residual Std. Error & 0.468 (df=20) & 0.495 (df=20) & 0.078 (df=20) & 0.078 (df=20) \\
 F Statistic & 74.954$^{***}$ (df=4; 20) & 72.563$^{***}$ (df=4; 20) & 74.954$^{***}$ (df=4; 20) & 72.563$^{***}$ (df=4; 20) \\
\hline
\hline \\[-1.8ex]
\textit{Note:} & \multicolumn{4}{r}{$^{*}$p$<$0.1; $^{**}$p$<$0.05; $^{***}$p$<$0.01} \\
\end{tabular}
\end{table}
\end{center}

Table 2 depicts the relationship between respondents' economic
evaluations and income growth under such experimental conditions. For
participants framed with yearly growth information (Treatment 0), the
regression results reaffirm the patterns observed in the previous
analysis. `RDI\_4' maintains a larger coefficient, underscoring its
influential role in shaping economic evaluations, and the associated
hypothesis tests validate the statistical significance of these
findings.

Notably, in the context of Treatment 1, where participants are exposed
to both yearly and cumulative growth information, the coefficients
exhibit a balanced distribution. Provided the additional information,
`RDI\_4' no longer emerges as a key factor influencing economic
evaluations, demonstrating a clear change in the previous pattern.
Although, their hypothesis tests show a p-value less than 0.001 for the
case where participants observe only yearly growth and a p-value of
0.156 for the case where participants observe both cumulative and yearly
growth, authors find that for individual level regressions, the ratio of
the weight participants place on the fourth year relative to the first
decreases substantially, from 4.92 to 1.11 (p \textless{} 0.001).

\begin{figure}
\hypertarget{fig:label}{%
\centering
\includegraphics[width=1\textwidth,height=1\textheight]{difcuma.png}
\caption{Treatment Effect: Cross-Section}\label{fig:label}
}
\end{figure}

Figure 6 displays the combinations of economic evaluations (\(economy\))
and Election-Year Real Disposable Income Growth (\(RDI_y\)) of the 25
hypothetical economies, in a cross-sectional manner. It's evident that
the treatment rewards (punishes) economic evaluations, particularly at
the highest (lowest) ends of (\(RDI_y\)). For combinations characterized
by less extreme election-year growth values, we anticipate a milder yet
discernible effect. For a chronological perspective on this phenomenon,
please refer to Figure 7, which displays the hypothetical election-year
growth trends alongside economic evaluations for both control and
treatment conditions.

\begin{figure}
\hypertarget{fig:label}{%
\centering
\includegraphics[width=1\textwidth,height=1\textheight]{difcum2.png}
\caption{Treatment Effect: Time-Series}\label{fig:label}
}
\end{figure}

The framing manipulation will involve presenting participants with real
economic data framed at either the yearly level (control condition) or
both the yearly and cumulative levels (treatment condition). This
manipulation will allow us to examine whether framing economic
information in a way that emphasizes both yearly and cumulative
performance helps participants to consider the entire term of an
incumbent, rather than just the election year. We expect that our
findings will align with the observations from the original experiment,
particularly for Hypotheses 1.

\begin{itemize}
\tightlist
\item
  Hypothesis 1: In the blind conditions, we expect that participants in
  the treatment condition will have more accurate and fair evaluations
  of the incumbent's economic performance than participants in the
  control condition. This is because framing the economic information in
  a way that emphasizes both yearly and cumulative performance will help
  them to consider the entire term of the incumbent, rather than just
  the election year.

  \begin{itemize}
  \tightlist
  \item
    In the control condition, participants will implicitly assign more
    weight to election-year growth. The first three coefficients will
    have small positive coefficients, while the fourth will be larger.
  \item
    In the treatment condition, participants will implicitly assign
    balanced weights to each of the four years.
  \end{itemize}
\end{itemize}

The affective manipulation will involve presenting participants with
real economic data for either an unidentified incumbent (blind
condition) or an identified incumbent (identified condition). This
manipulation will allow us to assess the impact of affective judgments
on participants' evaluations.

\begin{itemize}
\tightlist
\item
  Hypothesis 2: In the identified conditions, we expect that
  participants in the treatment condition will be less likely to be
  influenced by their affective judgments of the incumbent than
  participants in the control condition. This is because the additional
  information provided in the treatment condition will help them to make
  a more objective assessment of the incumbent's economic performance.

  \begin{itemize}
  \tightlist
  \item
    In the control condition, participants will implicitly assign more
    weight to election-year growth. The first three coefficients will
    have small positive coefficients, while the fourth will be larger.
    However, compared to the blind condition we expect more variance in
    the responses due to the influence of affective judgements.
  \item
    In the treatment condition, participants will implicitly assign
    balanced weights to each of the four years. However, compared to the
    control condition we expect less variance in the responses because
    of the mitigating effect of framing over the influence of affective
    judgements.
  \end{itemize}
\item
  Hypothesis 3: Finally, we expect that the treatment effect will be
  larger (smaller) among participants who hold weak (strong) positive or
  negative affective judgments of the incumbent. This backlash is
  because individuals with stronger affective judgments are more likely
  to rationalize or resist information when making evaluations, whereas
  those with weaker affective judgements may be more receptive to new
  information.

  \begin{itemize}
  \tightlist
  \item
    In the identified treatment condition, participants with weak
    (strong) affective judgements will implicitly assign more (less)
    balanced weights to each of the four years.
  \end{itemize}
\end{itemize}

\begin{center}
  \begin{table}[!htbp]
\centering
\caption{Real Economies Framing and Affective Manipulation}

\begin{tabular}{@{\extracolsep{2pt}}lcccc}
\\[-1.8ex]\hline
\hline \\[-1.8ex]
\\[-1.8ex] & \multicolumn{1}{c}{Treatment 0 - Blind} & \multicolumn{1}{c}{Treatment 0 - Identified} & \multicolumn{1}{c}{Treatment 1 - Blind} & \multicolumn{1}{c}{Treatment 1 - Identified}  \\
\\[-1.8ex] & (1) & (2) & (3) & (4) \\
\hline \\[-1.8ex]
 RDI\textsubscript{1} & -0.020$^{*}$ & 0.015$^{}$ & 0.026$^{}$ & 0.032$^{}$ \\
& (0.010) & (0.022) & (0.024) & (0.025) \\
 RDI\textsubscript{2} & 0.026$^{**}$ & 0.002$^{}$ & -0.005$^{}$ & -0.004$^{}$ \\
& (0.010) & (0.022) & (0.024) & (0.025) \\
 RDI\textsubscript{3} & 0.080$^{***}$ & 0.047$^{}$ & 0.009$^{}$ & 0.007$^{}$ \\
& (0.013) & (0.029) & (0.032) & (0.034) \\
 RDI\textsubscript{4} & 0.131$^{***}$ & 0.104$^{***}$ & 0.047$^{}$ & 0.038$^{}$ \\
& (0.015) & (0.033) & (0.037) & (0.038) \\
 const & 0.101$^{*}$ & 0.056$^{}$ & 0.171$^{}$ & 0.236$^{}$ \\
& (0.053) & (0.118) & (0.131) & (0.137) \\
\hline \\[-1.8ex]
 Observations & 17 & 17 & 17 & 17 \\
 $R^2$ & 0.916 & 0.587 & 0.255 & 0.245 \\
 Adjusted $R^2$ & 0.888 & 0.449 & 0.006 & -0.007 \\
 Residual Std. Error & 0.102 (df=12) & 0.226 (df=12) & 0.251 (df=12) & 0.262 (df=12) \\
 F Statistic & 32.822$^{***}$ (df=4; 12) & 4.256$^{**}$ (df=4; 12) & 1.025$^{}$ (df=4; 12) & 0.973$^{}$ (df=4; 12) \\
\hline
\hline \\[-1.8ex]
\textit{Note:} & \multicolumn{4}{r}{$^{*}$p$<$0.1; $^{**}$p$<$0.05; $^{***}$p$<$0.01} \\
\end{tabular}
\end{table}
\end{center}

For the control condition we expect the values to converge to the values
of the original experiment as displayed in Column (1) of Table 3. At a
lower election-year growth rate, the evaluations will cluster around the
lower values of economy, while at a higher election-year growth rate,
they will concentrate around the higher values. In the treatment
condition, at a lower cumulative growth rate, the evaluations will be
punished with respect to the evaluations in the control condition, and
at a higher cumulative growth rate, they will be rewarded with respect
the evaluations in the control condition. Figure 8 (left) displays the
extreme values while Figure 8 (right) displays the average effect around
the 33rd percentiles. These predictions are displayed with blue bars for
control and red for treatment conditions in Figure 8.

\begin{figure}
\hypertarget{fig:label}{%
\centering
\includegraphics[width=1\textwidth,height=1\textheight]{treat_lowthird.png}
\caption{Hypothetical Economies Treatment Effects}\label{fig:label}
}
\end{figure}

Building on Campello \& Zucco (2022), we introduce presidents' names
into the individual (\(RDI\)) plots to test the affective treatment
condition. With respect to the blind conditions, for the Identified
Control, we expect a larger effect on the evaluations caused by
participants affective judgements, while for the Identified Treatment a
smaller one (\(Yellow Predicted Economy\)). Results will have to be
detailed to consider the heterogeneity of participants and their
propensity to rationalize or update and embrace new or challenging
information. We expect the affective treatment to hold statistical
significance for values around the framing treatment, as displayed in
Figure 9. This outcome could provide compelling evidence about the
conditions when yearly and cumulative framing remain robust even when
participants can directly associate economic performance with a
president, effectively mitigating the end-bias and the influence of
prior affective judgments.

\begin{figure}
\hypertarget{fig:label}{%
\centering
\includegraphics[width=1\textwidth,height=1\textheight]{pred_low.png}
\caption{Real Economies Predicted Treatment Effects}\label{fig:label}
}
\end{figure}

\hypertarget{results-of-the-online-experiments-tbd}{%
\subsection{\texorpdfstring{Results of the Online Experiments
\emph{{[}TBD{]}}}{Results of the Online Experiments {[}TBD{]}}}\label{results-of-the-online-experiments-tbd}}

\hypertarget{implications-limitations-and-future-research-tbd}{%
\subsection{\texorpdfstring{Implications, Limitations, and Future
Research
\emph{{[}TBD{]}}}{Implications, Limitations, and Future Research {[}TBD{]}}}\label{implications-limitations-and-future-research-tbd}}

\hypertarget{discussion-and-conclusion-tbd}{%
\subsection{\texorpdfstring{Discussion and Conclusion
\emph{{[}TBD{]}}}{Discussion and Conclusion {[}TBD{]}}}\label{discussion-and-conclusion-tbd}}

In this study, we investigate voters' tendencies to emphasize
election-year economic performance and its significant implications for
democratic accountability. By introducing a variation in our
experimental design, we aimed to directly test the end-heuristic
explanation and the effects of voters' affective judgments. The results
confirm previous findings in the literature: voters' focus on the
election-year economy is not necessarily driven by an intentional choice
but rather by cognitive biases. Participants' evaluations became more
balanced when we provided them with information on cumulative
performance, highlighting the potential for a straightforward remedy to
mitigate the recency effects in voter decision-making. {[}Furthermore,
we contribute to the literature by demonstrating that cumulative framing
with identified incumbents can yield balanced evaluations consistently.
Specifically, balanced evaluations remain consistent for participants
with weak affective judgements but to a lesser extent for participants
with strong ties towards politicians which provides evidence of the
heterogeneity of participants and their responses to cumulative framing
and presidential identification.{]} These findings imply that changes in
how economic data are framed and presented by government agencies, the
news media, or candidates could influence voter behavior and enhance
democratic accountability.

The findings of this study shed light on the complex interplay between
framing, cognitive biases, and affective judgments in the context of
retrospective voting and democratic accountability. Our hypotheses have
provided valuable insights into how participants evaluate presidential
economic performance under different conditions. Hypothesis 1
demonstrated that participants tend to assign greater weight to
election-year growth when provided with yearly framing, reflecting the
influence of the end bias. This finding underscores the significance of
considering the timing of economic conditions during the electoral
process. Moreover, it revealed that when participants are presented with
both yearly and cumulative framing, their evaluations become more
balanced, suggesting that cumulative framing can mitigate the end bias
and encourage a more holistic assessment of economic performance.

{[}Building upon these insights, Hypothesis 2 explored the role of
affective judgments and presidential identification. It demonstrated
that participants' evaluations are influenced by their emotional ties to
political leaders, particularly when presidential names are included in
the economic data. This suggests that emotional judgments can have a
substantial impact on retrospective voting, potentially biasing
evaluations in favor of or against incumbents. Finally, Hypothesis 3
provided evidence of the heterogeneity of participants and their
responses to cumulative framing and presidential identification.
Balanced evaluations remain consistent for participants with weak
affective judgements but we capture a less discernible effect for
participants with strong ties towards politicians. This indicates that
the mitigation of the end bias and balanced assessments are robust
findings that hold contingent on participants' pre-existing affective
judgements. In conclusion, our study highlights the importance of
framing techniques, cognitive biases, and emotional judgments in shaping
voters' evaluations of presidential economic performance, contributing
to a deeper understanding of this intricate democratic process.{]}

The implications of our study extend beyond the realm of economic
evaluations during elections. We argue that this and other cognitive
biases and perceptions may have broader repercussions, affecting various
domains of incumbent performance assessment. The end heuristic's
influence could extend to evaluations of politicians' effectiveness in
areas beyond the economy, thereby posing a more general challenge to
democratic accountability. However, our results also provide hope, as
they suggest that relatively simple modifications in the information
context, such as emphasizing cumulative growth, may empower voters to
make more informed and balanced assessments of identified incumbents.
While implementing these changes in real-world settings poses
challenges, acknowledging the cognitive underpinnings of this phenomenon
opens the door to potential solutions that could enhance the functioning
of democratic systems and reduce the undue influence of election-year
economic fluctuations and affective judgments on electoral outcomes.

\singlespacing

\hypertarget{references}{%
\subsection{References}\label{references}}

\hypertarget{refs}{}
\begin{CSLReferences}{1}{0}
\leavevmode\vadjust pre{\hypertarget{ref-achenbartels_2004}{}}%
Achen, C. H., \& Bartels, L. M. (2004). \emph{Musical {Chairs}:
{Pocketbook Voting} and the {Limits} of {Democratic Accountability}}.

\leavevmode\vadjust pre{\hypertarget{ref-bartels_1996}{}}%
Bartels, L. M. (1996). Uninformed {Votes}: {Information Effects} in
{Presidential Elections}. \emph{American Journal of Political Science},
\emph{40}(1), 194--230. \url{https://doi.org/10.2307/2111700}

\leavevmode\vadjust pre{\hypertarget{ref-Baumgarten_1997}{}}%
Baumgartner, H., Sujan, M., \& Padgett, D. (1997). Patterns of
{Affective Reactions} to {Advertisements}: {The Integration} of
{Moment-to-Moment Responses} into {Overall Judgments}. \emph{Journal of
Marketing Research}, \emph{Vol. XXXIV}.

\leavevmode\vadjust pre{\hypertarget{ref-camzuvol_2020}{}}%
Campello, D., \& Zucco, C. (2020). \emph{The {Volatility Curse}:
{Exogenous Shocks} and {Representation} in {Resource-Rich Democracies}}.
{Cambridge University Press}.
\url{https://doi.org/10.1017/9781108894975}

\leavevmode\vadjust pre{\hypertarget{ref-CampelloZucco_CommodityPrices2022}{}}%
Campello, D., \& Zucco, C. (2022). \emph{Commodity {Prices}, {Relative
Performance}, and {Misattribution} of {Responsibility} for the
{Economy}}.

\leavevmode\vadjust pre{\hypertarget{ref-ohpp_chong_2013}{}}%
Chong, D. (2013). Degrees of {Rationality} in {Politics}. In L. Huddy,
D. O. Sears, \& J. S. Levy (Eds.), \emph{The {Oxford Handbook} of
{Political Psychology}} (p. 0). {Oxford University Press}.
\url{https://doi.org/10.1093/oxfordhb/9780199760107.013.0004}

\leavevmode\vadjust pre{\hypertarget{ref-downs_1957}{}}%
Downs, A. (1957). \emph{An {Economic Theory} of {Democracy}}. {Harper}.

\leavevmode\vadjust pre{\hypertarget{ref-druckman_2001}{}}%
Druckman, J. N. (2001a). On the {Limits} of {Framing Effects}: {Who Can
Frame}? \emph{The Journal of Politics}, \emph{63}(4), 1041--1066.
\url{https://www.jstor.org/stable/2691806}

\leavevmode\vadjust pre{\hypertarget{ref-druckman_2001b}{}}%
Druckman, J. N. (2001b). The {Implications} of {Framing Effects} for
{Citizen Competence}. \emph{Political Behavior}, \emph{23}(3), 225--256.
\url{https://doi.org/10.1023/A:1015006907312}

\leavevmode\vadjust pre{\hypertarget{ref-fair_1978}{}}%
Fair, R. C. (1978). The {Effect} of {Economic Events} on {Votes} for
{President}. \emph{The Review of Economics and Statistics},
\emph{60}(2), 159. \url{https://doi.org/10.2307/1924969}

\leavevmode\vadjust pre{\hypertarget{ref-fiorina_1981}{}}%
Fiorina, M. P. (1981). \emph{Retrospective {Voting} in {American
National Elections}}. {Yale University Press}.

\leavevmode\vadjust pre{\hypertarget{ref-Goffman_1986}{}}%
Goffman, E. (1986). \emph{Frame analysis: An essay on the organization
of experience} (Northeastern University Press ed). {Northeastern
University Press}.

\leavevmode\vadjust pre{\hypertarget{ref-HealyLenz_2014}{}}%
Healy, A., \& Lenz, G. S. (2014). Substituting the {End} for the
{Whole}: {Why Voters Respond Primarily} to the {Election-Year Economy}.
\emph{American Journal of Political Science}, \emph{58}(1), 31--47.
\url{https://doi.org/10.1111/ajps.12053}

\leavevmode\vadjust pre{\hypertarget{ref-healmal_2013}{}}%
Healy, A., \& Malhotra, N. (2013). Retrospective {Voting Reconsidered}.
\emph{Annual Review of Political Science}, \emph{16}(1), 285--306.
\url{https://doi.org/10.1146/annurev-polisci-032211-212920}

\leavevmode\vadjust pre{\hypertarget{ref-huhilenz_2012}{}}%
Huber, G. A., Hill, S. J., \& Lenz, G. S. (2012). Sources of {Bias} in
{Retrospective Decision Making}: {Experimental Evidence} on {Voters}'
{Limitations} in {Controlling Incumbents}. \emph{American Political
Science Review}, \emph{106}(4), 720--741.
\url{https://doi.org/10.1017/S0003055412000391}

\leavevmode\vadjust pre{\hypertarget{ref-kahn_2003}{}}%
Kahneman, D. (2003). Maps of {Bounded Rationality}: {Psychology} for
{Behavioral Economics}. \emph{American Economic Review}, \emph{93}(5),
1449--1475. \url{https://doi.org/10.1257/000282803322655392}

\leavevmode\vadjust pre{\hypertarget{ref-kahn_2011}{}}%
Kahneman, D. (2011). \emph{Thinking, {Fast} and {Slow}}. {Farrar, Straus
and Giroux}.

\leavevmode\vadjust pre{\hypertarget{ref-kahntev_1979}{}}%
Kahneman, D., \& Tversky, A. (1979). Prospect {Theory}: {An Analysis} of
{Decision} under {Risk}. \emph{Econometrica}, \emph{47}(2), 263--291.
\url{https://doi.org/10.2307/1914185}

\leavevmode\vadjust pre{\hypertarget{ref-key_1966}{}}%
Key, V. O. J. (1966). \emph{The {Responsible Electorate}: {Rationality}
in {Presidential Voting}, 1936{\textendash}1960}. {Belknap Press}.

\leavevmode\vadjust pre{\hypertarget{ref-kramer_1971}{}}%
Kramer, G. H. (1971). Short-{Term Fluctuations} in {U}.{S}. {Voting
Behavior}, 1896{\textendash}1964. \emph{American Political Science
Review}, \emph{65}(1), 131--143. \url{https://doi.org/10.2307/1955049}

\leavevmode\vadjust pre{\hypertarget{ref-kukquirk_2000}{}}%
Kuklinski, J. H., \& Quirk, P. J. (2000). Reconsidering the {Rational
Public}: {Cognition}, {Heuristics}, and {Mass Opinion}. In A. Lupia, M.
D. McCubbins, \& S. L. Popkin (Eds.), \emph{Elements of {Reason}:
{Cognition}, {Choice}, and the {Bounds} of {Rationality}} (pp.
153--182). {Cambridge University Press}.
\url{https://doi.org/10.1017/CBO9780511805813.008}

\leavevmode\vadjust pre{\hypertarget{ref-Lakoff_2014}{}}%
Lakoff, G. (2014). \emph{The all-new don't think of an elephant! Know
your values and frame the debate}. {Chelsea Green Publishing}.

\leavevmode\vadjust pre{\hypertarget{ref-redlau_1997}{}}%
Lau, R. R., \& Redlawsk, D. P. (1997). {Voting correctly}.
\emph{American Political Science Review}, \emph{91}(3), 585--598.
\url{https://doi.org/10.2307/2952076}

\leavevmode\vadjust pre{\hypertarget{ref-redlau_2001}{}}%
Lau, R. R., \& Redlawsk, D. P. (2001). {Advantages and disadvantages of
cognitive heuristics in political decision making}. \emph{American
journal of political science}, \emph{45}(4), 951--971.
\url{https://doi.org/10.2307/2669334}

\leavevmode\vadjust pre{\hypertarget{ref-taberlodge_2005}{}}%
Lodge, M., \& Taber, C. S. (2005). The {Automaticity} of {Affect} for
{Political Leaders}, {Groups}, and {Issues}: {An Experimental Test} of
the {Hot Cognition Hypothesis}. \emph{Political Psychology},
\emph{26}(3), 455--482.
\url{https://doi.org/10.1111/j.1467-9221.2005.00426.x}

\leavevmode\vadjust pre{\hypertarget{ref-taberlodge_2013}{}}%
Lodge, M., \& Taber, C. S. (2013). \emph{The {Rationalizing Voter}}.
{Cambridge University Press}.

\leavevmode\vadjust pre{\hypertarget{ref-Neuman_2000}{}}%
Marcus, G. E., Neuman, W. R., \& MacKuen, M. (2000). \emph{Affective
intelligence and political judgment}. {University of Chicago Press}.

\leavevmode\vadjust pre{\hypertarget{ref-McCombs_1972}{}}%
McCombs, M. E., \& Shaw, D. L. (1972). The {Agenda-Setting Function} of
{Mass Media}. \emph{The Public Opinion Quarterly}, \emph{36}(2),
176--187. \url{https://www.jstor.org/stable/2747787}

\leavevmode\vadjust pre{\hypertarget{ref-vonneumorg_1947}{}}%
Neumann, J. von, \& Morgenstern, O. (2007). \emph{Theory of {Games} and
{Economic Behavior}: 60th {Anniversary Commemorative Edition}}.
{Princeton University Press}.

\leavevmode\vadjust pre{\hypertarget{ref-redkahn_1996}{}}%
Redelmeier, D. A., \& Kahneman, D. (1996). Patients' memories of painful
medical treatments: Real-time and retrospective evaluations of two
minimally invasive procedures. \emph{Pain}, \emph{66}(1), 3--8.
\url{https://doi.org/10.1016/0304-3959(96)02994-6}

\leavevmode\vadjust pre{\hypertarget{ref-ohpp_redlau_2013}{}}%
Redlawsk, D. P., \& Lau, R. R. (2013). Behavioral {Decision-Making}. In
L. Huddy, D. O. Sears, \& J. S. Levy (Eds.), \emph{The {Oxford Handbook}
of {Political Psychology}} (p. 0). {Oxford University Press}.
\url{https://doi.org/10.1093/oxfordhb/9780199760107.013.0005}

\leavevmode\vadjust pre{\hypertarget{ref-simon_1957}{}}%
Simon, H. A. (1957). \emph{Models of {Man}: {Social} and {Rational};
{Mathematical Essays} on {Rational Human Behavior} in {Society
Setting}}. {Wiley}.

\leavevmode\vadjust pre{\hypertarget{ref-Sniderman_2004}{}}%
Sniderman, P. M., \& Theriault, S. M. (2004). \emph{The {Structure} of
{Political Argument} and the {Logic} of {Issue Framing}}.

\leavevmode\vadjust pre{\hypertarget{ref-ohpp_taberlodge_2013}{}}%
Taber, C. S., \& Young, E. (2013). Political {Information Processing}.
In L. Huddy, D. O. Sears, \& J. S. Levy (Eds.), \emph{The {Oxford
Handbook} of {Political Psychology}} (p. 0). {Oxford University Press}.
\url{https://doi.org/10.1093/oxfordhb/9780199760107.013.0017}

\leavevmode\vadjust pre{\hypertarget{ref-tufte_1978}{}}%
Tufte, E. R. (1978). \emph{Political control of the economy} (1.
Paperback ed). {Princeton Univ. Press}.

\leavevmode\vadjust pre{\hypertarget{ref-tvekahn_1974}{}}%
Tversky, A., \& Kahneman, D. (1974). Judgment under uncertainty:
{Heuristics} and biases. \emph{Science}, \emph{185}(4157), 1124--1131.
\url{https://doi.org/10.1126/science.185.4157.1124}

\end{CSLReferences}

\newpage

\hypertarget{annex}{%
\subsection{Annex}\label{annex}}

\hypertarget{github}{%
\subsubsection{Github}\label{github}}

Online version can be accessed at this
\href{https://danybonfil.github.io/start/}{\textbf{Github Repository}}

\hypertarget{survey-details}{%
\subsubsection{Survey Details}\label{survey-details}}

Survey can be accessed at this
\href{https://survey.fgv.br/jfe/form/SV_5jUiMOHHgM1t0DY}{\textbf{Qualtrics
Link}}

\begin{figure}
\hypertarget{fig:label}{%
\centering
\includegraphics[width=1\textwidth,height=1\textheight]{flow_survey.png}
\caption{Survey Flow}\label{fig:label}
}
\end{figure}

\begin{figure}
\hypertarget{fig:label}{%
\centering
\includegraphics[width=1\textwidth,height=1\textheight]{start_survey.png}
\caption{Start Survey}\label{fig:label}
}
\end{figure}

\begin{figure}
\hypertarget{fig:label}{%
\centering
\includegraphics[width=1\textwidth,height=1\textheight]{control_blind.png}
\caption{Control Blind}\label{fig:label}
}
\end{figure}

\begin{figure}
\hypertarget{fig:label}{%
\centering
\includegraphics[width=1\textwidth,height=1\textheight]{control_identified.png}
\caption{Control Identified}\label{fig:label}
}
\end{figure}

\begin{figure}
\hypertarget{fig:label}{%
\centering
\includegraphics[width=1\textwidth,height=1\textheight]{treat_blind.png}
\caption{Treatment Blind}\label{fig:label}
}
\end{figure}

\begin{figure}
\hypertarget{fig:label}{%
\centering
\includegraphics[width=1\textwidth,height=1\textheight]{treat_identified.png}
\caption{Treatment Identified}\label{fig:label}
}
\end{figure}

\begin{figure}
\hypertarget{fig:label}{%
\centering
\includegraphics[width=1\textwidth,height=1\textheight]{end_survey.png}
\caption{End Survey}\label{fig:label}
}
\end{figure}

\end{document}
